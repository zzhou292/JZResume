%-------------------------------------------------------------------------------
%	SECTION TITLE
%-------------------------------------------------------------------------------
\cvsection{Work Experience}


%-------------------------------------------------------------------------------
%	CONTENT
%-------------------------------------------------------------------------------
\begin{cventries}
	
  \cventry
	{Graduate Research Assistant} % Job title
	{Simulation-Based Engineering Lab at UW-Madison} % Organisation
	{Madison, WI} % Location
	{Jan 2021 - Present} % Date(s)
	{
		\begin{cvitems} % Description(s) of tasks/responsibilities
			\item {Advisor: Professor Dan Negrut.}
			\item {Autonomous vehicle coordination and simulation of vehicle dynamics, leveraging simulation fidelity and real-time performance for Human-In-The-Loop(HIL) and Software-In-The-Loop(SIL) applications. Head developer of chrono::HIL, a submodule of Project Chrono (\url{https://projectchrono.org/}) to provide HIL, real-time simulation support for traffic scenarios and vehicle dynamics. chrono:HIL provides flexible simulator hardware coupling capabilities, distributed simulation support, soft real-time simulation support, and multiple vehicle dynamic models. Integrating sensor (lidar/radar/camera sensor) simulation using chrono::sensor into traffic scenarios to assist the development of autonomous vehicle control policies. Providing simulation support for human-factor research conducted by Cognitive Systems Laboratory at UW-Madison. Funded by National Science Foundation OAC2209791.}
			\item {Extraterrestrial rover and robot mechanical component simulation. Applications/development/validation of SCM (Soil Contact Model), SPH, and DEM deformable terrain. Developer of the VIPER lunar rover model and the Curiosity mars rover model in the chrono::robot module. Integrating sensor simulation support to provide Lidar/Radar perception data in harsh lunar environment. Funded by NASA to support 2023 VIPER lunar mission.}
		\end{cvitems}
	}
		
	
%--------------------------------------------------------

  \cventry
	{Undergraduate Research Assistant} % Job title
	{Simulation-Based Engineering Lab at UW-Madison} % Organisation
	{Madison, WI} % Location
	{Jun 2020 - Dec 2020} % Date(s)
	{
		\begin{cvitems} % Description(s) of tasks/responsibilities
			\item {Advisor: Professor Dan Negrut.}
			\item {Development and validation of chrono::granular (later renamed as chrono::gpu), a CUDA solver for granular dynamics. chrono::granular can be used to simulate homogeneous granular material; applications include granular material properties testing and deformable terrian for off-road vehicle research.}
			\item {Development of synchronization functionalities in chrono::synchrono - MPI and DDS interfaces of chrono::vehicles; utilization of parallel computing for real-time performance.}
		\end{cvitems}
	}


% --------------------------------------------------------

  \cventry
	{Undergraduate Research Assistant} % Job title
	{Human Computer Interaction Lab at UW-Madison} % Organisation
	{Madison, WI} % Location
	{Sep 2019 - May 2020} % Date(s)
	{
	  \begin{cvitems} % Description(s) of tasks/responsibilities
		\item {Advisor: Professor Bilge Mutlu.}
		\item {Developing a QR Marker object tracking program based on OpenCV in C++. The program helps educational robots to identify objects and their movements in order to facilitate human-computer interaction.}
		\item {Designing and developing of simulation environment for robot localization algorithm using ROS2. The simulation environment allows a Turtlebot model to follow certain trajectories in an indoor environment relying purely on QR codes identified by the machine learning algorithm.}
		\item {Creating of the CAD models for robot's parts  using Solidworks and 3D printing software.}
	  \end{cvitems}
}


%---------------------------------------------------------
  \cventry
    {Software Engineering Intern \& Embedded System Engineering Intern} % Job title
    {Alstom} % Organisation
    {Melbourne, FL} % Location
    {May 2019 - Aug 2019} % Date(s)
    {
      \begin{cvitems} % Description(s) of tasks/responsibilities
        \item {Cooperating with Alstom's System Validation Team to perform system tests and review code (primarily in C++ and Python) on Alstom DAU (Data Acquisition Unit), a vital wayside component of the Alstom's Automatic Railway Signaling System; Debugging lower-level program, scanning and hacking the Apache server installed to search for possible bugs which may lead to the fatal crash of the system.}
        \item {Developing a C++ testing program for Alstom's Wayside Linux-Based Core ACE board to meet Hardware Serial Test Specifications including multi-CPU communication (based on C++ socket), I2C, UART, SPI, onboard GPIO connection, Watchdog Timer, and other hardware checks. The testing program includes both lower-level hardware programming (hardware read and write interfaces, UDP socket communication designed for multi-CPU connection, file read and write operations used to check the functionalities of FRAM, Flash Memory, and eMMC) and higher-level software programming (user Interface, comparison algorithm used to determine whether the actual result matches original expectations).}
        \item {Participating in technical reviews, technology transfer meetings, and code reviews. Learning the coding standard in Alstom.}
        \item {Learning concepts and architectures of the modern autonomous railway system and contributing to the design of the system.}
      \end{cvitems}
    }

%---------------------------------------------------------
\end{cventries}
